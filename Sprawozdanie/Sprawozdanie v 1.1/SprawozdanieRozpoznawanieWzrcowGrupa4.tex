%% 
%% Copyright 2019-2020 Elsevier Ltd
%% 
%% This file is part of the 'CAS Bundle'.
%% --------------------------------------
%% 
%% It may be distributed under the conditions of the LaTeX Project Public
%% License, either version 1.2 of this license or (at your option) any
%% later version.  The latest version of this license is in
%%    http://www.latex-project.org/lppl.txt
%% and version 1.2 or later is part of all distributions of LaTeX
%% version 1999/12/01 or later.
%% 
%% The list of all files belonging to the 'CAS Bundle' is
%% given in the file `manifest.txt'.
%% 
%% Template article for cas-dc documentclass for 
%% double column output.

\documentclass[a4paper,fleqn]{cas-dc}
\usepackage[authoryear,longnamesfirst]{natbib}


\begin{document}
\let\WriteBookmarks\relax
\def\floatpagepagefraction{1}
\def\textpagefraction{.001}

% Krótki tytuł
\shorttitle{Analiza zbioru pacjętów z SM chorych na COVID-19 modelem One-CLass SVM}

% Skrót autorów
\shortauthors{}

% Tytuł
\title [mode = title]{Analiza zbioru danych "Patient-level dataset to study the effect of COVID-19 in people with Multiple Sclerosis" z wykorzystaniem nadzorowanego algorytmu uczenia maszynowego One-Class SVM(Support Vector Machines)}                      

% Autorzy
\author{Anna Mrozek, Bartosz Panek \,i\, Przemysław Jura  }

% Streszczenie
\renewcommand{\abstractname}{STRESZCZENIE}
\begin{abstract}
Artykuł analizuje zbiór danych pacjentów ze stwardnieniem rozsianym (SM), którzy przeszli COVID-19, przy użyciu nadzorowanego algorytmu One-Class SVM. Celem jest identyfikacja przypadków o zwiększonym ryzyku ciężkiego przebiegu infekcji. Umożliwi to zidentyfikowanie charakterystycznych cech pacjentów bardziej narażonych na powikłania.
\end{abstract}

% Słowa klucze
\renewcommand{\abstractname}{STRESZCZENIE}
\begin{keywords}
One-Class SVM \sep COVID-19 
\end{keywords}

\maketitle

\section{Wprowadzenie}
Pandemia COVID-19 miała znaczący wpływ na osoby z chorobami przewlekłymi, w tym na pacjentów ze stwardnieniem rozsianym (SM). SM jest przewlekłą chorobą autoimmunologiczną, która wpływa na układ nerwowy i może prowadzić do trwałego uszkodzenia neuronów oraz ograniczenia funkcji motorycznych i poznawczych. Ze względu na charakter choroby i stosowane terapie immunosupresyjne, pacjenci z SM mogą być bardziej narażeni na cięższy przebieg infekcji wirusowych, w tym COVID-19. Analiza danych od pacjentów z SM, którzy przeszli COVID-19, może dostarczyć ważnych informacji na temat ryzyka powikłań i zidentyfikować czynniki przyczyniające się do poważniejszych objawów, co ostatecznie może pomóc w lepszym monitorowaniu i opiece nad tą grupą pacjentów.




\end{document}

